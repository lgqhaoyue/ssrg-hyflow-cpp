\documentclass[12pt,english]{report}

\setlength{\textwidth}{6.5in}
\setlength{\textheight}{8.5in}
\setlength{\evensidemargin}{0in}
\setlength{\oddsidemargin}{0in}
\setlength{\topmargin}{0in}
\setlength{\parindent}{0pt}
\setlength{\parskip}{0.1in}

\setcounter{secnumdepth}{3}
\setcounter{tocdepth}{3}

%% A simple dot to overcome graphicx limitations
\newcommand{\lyxdot}{.}

% Uncomment for double-spaced document.
\renewcommand{\baselinestretch}{1}

% \usepackage{epsf}
\usepackage{graphicx}
\usepackage{listings}\lstset{
language=Java,                		% choose the language of the code
basicstyle=\footnotesize,       % the size of the fonts that are used for the code
numbers=left,                   % where to put the line-numbers
numberstyle=\footnotesize,      % the size of the fonts that are used for the line-numbers
stepnumber=1,                   % the step between two line-numbers. If it's 1 each line will be numbered
numbersep=5pt,                  % how far the line-numbers are from the code
showspaces=false,               % show spaces adding particular underscores
showstringspaces=false,         % underline spaces within strings
showtabs=false,                 % show tabs within strings adding particular underscores
%frame=single,	                % adds a frame around the code
tabsize=4,		                % sets default tabsize to 2 spaces
captionpos=b,                   % sets the caption-position to bottom
breaklines=true,                % sets automatic line breaking
breakatwhitespace=false,        % sets if automatic breaks should only happen at whitespace
}
\usepackage{subfigure}

\makeatother
\usepackage{graphicx}
\usepackage{epstopdf}
\usepackage{babel}
\usepackage{hyperref}
\hypersetup{
	colorlinks=true,		% false: boxed links; true: colored links
	linkcolor=black,          % color of internal links
    citecolor=black,        % color of links to bibliography
    filecolor=black,      % color of file links
    urlcolor=black           % color of external links
}
\usepackage{rotating}
\usepackage{comment}
%\usepackage{bbding}
\usepackage{threeparttable}




\begin{document}

\thispagestyle{empty}
\pagenumbering{roman}
\begin{center}

% TITLE
{\Large 
HyflowCPP : Distributed Transaction Memory framework for C++
}

\vfill

Sudhanshu Mishra

\vfill

Thesis submitted to the Faculty of the \\
Virginia Polytechnic Institute and State University \\
in partial fulfillment of the requirements for the degree of

\vfill

Master of Science \\
in \\
Computer Engineering


\vfill

Binoy Ravindran, Chair \\
Robert P. Broadwater \\
Mark Jones


\vfill

December 7, 2012 \\
Blacksburg, Virginia

\vfill

Keywords: Distributed Software Transaction Memory, Transactional Framework, C++, Concurrency
\\
Copyright 2012, Sudhanshu Mishra

\end{center}

\pagebreak

\thispagestyle{empty}
\begin{center}

{\large
HyflowCPP : Distributed Transaction Memory framework for C++
}

\vfill

Sudhanshu Mishra

\vfill

(ABSTRACT)

\vfill

\end{center}

To Be Added




\vfill

% GRANT INFORMATION

% This work was partially supported by the US National Science Foundation.


\pagebreak

% Dedication and Acknowledgments are both optional
\chapter*{Dedication}

\begin{center}
I dedicate this thesis to my family and friends.

\textit{Without their support this would not have been possible}

\end{center}


\chapter*{Acknowledgments}

I would like to thank my advisor, Dr. Binoy Ravindran, for his 
help and guidance on both technical and personal 
topics. It has been an honor to work under him and I am highly thankful
to him for his trust in me.

I would also like to thank Dr. Robert Broadwater and Dr. Mark Jones,
for serving on my committee and providing their valuable feedback
and direction. In addition, I would like to thank all of my colleagues
at the Systems Software Research lab. I would particularly like to thank
Alex Turcu, Mohd. Saad and Aditya Dhoke for their support and encouragement.
It was a pleasure to work with them and perform interesting research in area 
of Distributed Transactional Memory.

Finally, I would like to thank my family and friends for all the
love and support they have given me, without which this thesis would 
not have been possible.

All figures in thesis are the work of the author, unless specified otherwise.

\tableofcontents
\pagebreak

\listoffigures
\pagebreak

%\listofalgorithms
%\pagebreak

\listoftables
\pagebreak

%\printnomenclature
%\pagebreak

\pagenumbering{arabic}
\pagestyle{myheadings}


%%%%%%%%%%%%%%%%%%%%%%%%%%%%%%%%%%%%%%%%%%%%%%%%%%%
%																									%
%							CHAPTER 1	:	INTRODUCTION						%
%																									%
%%%%%%%%%%%%%%%%%%%%%%%%%%%%%%%%%%%%%%%%%%%%%%%%%%%
\chapter{Introduction}\label{chap:intro}
\markright{Chapter~\ref{chap:intro}.
Introduction
\hfill}

DSTM provides an alternative to classical lock-based distributed concurrency control for achieving concurrency in a networked environment. 

\section{Thesis Contribution}

To be Added

\section{Thesis Organization}

The rest of the thesis is organized as follows: Chapter~\ref{chap:relWork} overviews past and related work in the DSTM space, and contrasts them with the thesis's problem space. Chapter~\ref{chap:progInterface} illustrates the programming model required to develop benchmarks in our framework. Chapter~\ref{chap:sysArch} describes our framework architecture and interaction between different components. Chapter~\ref{chap:algorithm} explains the TFA algorithm and its adoption in different transactional models.We report our experimental results in Chapter~\ref{chap:expResults}. Finally, we conclude the thesis in Chapter~\ref{chap:conclusion}.

%%%%%%%%%%%%%%%%%%%%%%%%%%%%%%%%%%%%%%%%%%%%%%%%%%%
%																									%
%							CHAPTER 2	:	Related work						%
%																									%
%%%%%%%%%%%%%%%%%%%%%%%%%%%%%%%%%%%%%%%%%%%%%%%%%%%
\chapter{Related Work}\label{chap:relWork}
\markright{Chapter~\ref{chap:relWork}.
Related Work
\hfill}

Research in DSTM space have been relatively small and recent in comparison to STM. Manassiev~\cite{Manassiev:2006:EDV:1122971.1123002} work based on Distributed Multiversioning(DMV) can be considered one of first paper on DSTM published in 2006. It
introduced a novel page-level distributed concurrency control algorithm, called Distributed Multiversioning(DMV), which automatically detected and resolved conflicts caused by data races for distributed transactions accessing shared in-memory data structures. DMV’s key feature was to exploit the distributed data versions in order to avoid read-write conflicts. Later,
In 2007 Herily~\cite{Herilhy:2007:BallisticProtocal} described the problem of implementing transactional memory in a network of nodes and prosposed a newcache-coherence protocol,
called the Ballistic protocol. In this protocol nodes are organized as clusters at different levels. One node in each cluster is chosen to act as leader for this cluster when communicating with clusters at different levels. When a transaction requests an object, the request rises in the hierarchy, probing leaders at increasing levels until the request encounters a downward link. When the request finds such a link, it descends, following a chain of links down to the cached copy of the object. But both of these systems had scalability issues to which later Cluster-STM~\cite{Bocchino:2008:STM:1345206.1345242} tried to resolve using partitioned global address space (PGAS) model. In later years DSTM systems were developed around various properties. These works can be classified on basis of mobility of transactions(control flow/Data flow), number of a active copies of transactional objects(single copy/replica), valid versions of objects(single version/multiverson) or transactional properties like memory consistency, serializability and opacity etc. For our discussion purpose we will divide the work in DSTM space based on serializibility property:
\begin{enumerate}
\item Serializable DSTM implementations
\item Non-Serializable DSTM implementations 
\end{enumerate}


\section{Serializable DSTM implementations}

Decent STM, GenRSTM, 
To be Added

\section{Non-Serializable DSTM implementations}

To be Added

%%%%%%%%%%%%%%%%%%%%%%%%%%%%%%%%%%%%%%%%%%%%%%%%%%%
%																									%
%							CHAPTER 3	:	Programming Interface						%
%																									%
%%%%%%%%%%%%%%%%%%%%%%%%%%%%%%%%%%%%%%%%%%%%%%%%%%%
\chapter{Programming Interface}\label{chap:progInterface}
\markright{Chapter~\ref{chap:progInterface}.
Programming Interface
\hfill}

To Be Added


%%%%%%%%%%%%%%%%%%%%%%%%%%%%%%%%%%%%%%%%%%%%%%%%%%%
%																									%
%							CHAPTER 4	:	System Architecture						%
%																									%
%%%%%%%%%%%%%%%%%%%%%%%%%%%%%%%%%%%%%%%%%%%%%%%%%%%

\chapter{System Architecture}\label{chap:sysArch}
\markright{Chapter~\ref{chap:sysArch}.
System Architecture
\hfill}

To be Added

%%%%%%%%%%%%%%%%%%%%%%%%%%%%%%%%%%%%%%%%%%%%%%%%%%%
%																									%
%							CHAPTER 5	:	Algorithms						%
%																									%
%%%%%%%%%%%%%%%%%%%%%%%%%%%%%%%%%%%%%%%%%%%%%%%%%%%
\chapter{Algorithm}\label{chap:algorithm}
\markright{Chapter~\ref{chap:algorithm}.
Algorithm
\hfill}

To be Added

%%%%%%%%%%%%%%%%%%%%%%%%%%%%%%%%%%%%%%%%%%%%%%%%%%%
%																									%
%							CHAPTER 6	:	Experiments						%
%																									%
%%%%%%%%%%%%%%%%%%%%%%%%%%%%%%%%%%%%%%%%%%%%%%%%%%%
\chapter{Experimental Results \& Evaluation}\label{chap:expResults}
\markright{Chapter~\ref{chap:expResults}.
Experimental Results \& Evaluation
\hfill}

In this chapter, the performance of HyflowCPP is compared against other Java STMs using micro-benchmarks and macro-benchmarks.

\section{Test Environment}

To Be Added

\section{Micro-Benchmarks}

To Be Added 

\subsection{Linked List\label{sub:Linked-List}}

To Be Added 

\subsection{Skip List}

To Be Added 

\subsection{Red-Black Tree}

To Be Added 

\section{Macro Benchmarks}

In this section, we evaluate the performance under macro-benchmarks including a Bank application and five applications from the STAMP benchmark suite~\cite{caominh:stamp:iiswc:2008} (Vacation, KMeans, Genome,
Labyrinth and Intruder).

\subsection{Bank}

To Be Added

\subsection{Vacation}

To Be Added

\subsection{Loan}

To Be Added

\section{Summary}


%%%%%%%%%%%%%%%%%%%%%%%%%%%%%%%%%%%%%%%%%%%%%%%%%%%
%																									%
%							CHAPTER 7	:	Conclusion						%
%																									%
%%%%%%%%%%%%%%%%%%%%%%%%%%%%%%%%%%%%%%%%%%%%%%%%%%%
\chapter{Conclusion and Future Work}\label{chap:conclusion}
\markright{Chapter~\ref{chap:conclusion}.
Conclusion and Future Work
\hfill}

To be Added

\section{Future Work}

Several directions exist for future work. These include the following:

\newpage
\markright{Bibliography \hfill}

\bibliographystyle{abbrv}
\addcontentsline{toc}{chapter}{Bibliography}
\bibliography{BibTex/all}

\end{document}
